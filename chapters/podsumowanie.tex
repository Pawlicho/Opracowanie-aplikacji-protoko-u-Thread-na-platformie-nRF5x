\chapter*{Podsumowanie}

Podczas realizacji pracy, wykonano kolejne kroki:
\begin{enumerate}
    \item Zapoznano się z charakterystyką stosu Thread i podstawowymi mechanizmami sieci Thread.
    \item Zaprojektowano prototypowy system automatyki domowej, dokonując rozdziału funkcjonalności na poszczególne komponenty.
    \item Zaznajomiono się z platformą sprzętową Nordic Semiconductor nRF52833 oraz narzędziami dostarczonymi przez producenta, jak również z otwartą, wolną, implementacją stosu Thread Openthread.
    \item Stworzono w pełni funkcjonalną sieć Thread złożoną z 5 urządzeń platformy nRF52833. Uwzględniono mechanizm bezpieczeństwa Commissioning oraz zagwarantowano połączenie sieci z domeny Thread z zewnętrzną siecią Ethernet poprzez uruchomienie aplikacji Rutera Brzegowego.
    \item Zaimplementowano aplikację zaproponowanego systemu, według poczynionych założeń. Zapewniono komunikację klient-serwer z wykorzystaniem protokołu CoAP. Zrealizowano pozostałe komponenty takie jak Baza Danych oraz Web GUI, po wcześniejszym zapoznaniu się z wykorzystywanymi technologiami.
    \item Uruchomiono sieć Thread i sprawdzono połączenia między węzłami.
    \item Wystartowano aplikację systemu automatyki domowej oraz zweryfikowano poprawność działania poprzez analizę wykresów wygenerowanych za pośrednictwem narzędzia Grafana.
    \item Zbadano wykorzystanie zasobów pamięci ulotnej oraz nieulotnej przez protokół Thread, oraz jego funkcjonalności dla platformy nRF52833 z implementacją OpenThread.
\end{enumerate}

W związku z charakterem pracy, której autor samodzielnie dokonywał implementacji oraz ograniczeniem czasowym przeznaczonym na realizację, zaprojektowany system ma charakter prototypu. W celu polepszenia funkcjonalności systemu oraz przygotowania go do wdrożenia,
zaprojektowany system mógłby zostać wzbogacony o dodatkowe funkcjonalności, które wymieniono poniżej:
\begin{itemize}
    \item Wprowadzenie dodatkowych mechanizmów bezpieczeństwa poprzez zastosowanie warstwy DTLS dla gniazd UDP, wykorzystywanych do transmisji zapytań i odpowiedzi CoAP. 
    \item Zmiana mechanizmu On-mesh Commissioning na External Commissioning, wykorzystując implementację  OpenThread. Stworzenie aplikacji mobilnej, który umożliwiałaby uwierzytelnienie węzłów Joiner poprzez zbliżenie anten NFC (ang. \textit{Near-Field Communication}).
    \item Zamiana prototypowych algorytmów regulacji po stronie System Controllera na algorytmy wykorzystywane w automatyce przemysłowej, takie jak np. PID.
    \item Wprowadzenie do urządzeń końcowych HEATER oraz DIMMER fizycznych czujników oraz układów regulacji, w celu weryfikacji działania systemu w rzeczywistym środowisku domowym oraz dostosowania algorytmów System Controllera do istniejących warunków budynkowych.
    \item Przeniesienie aplikacji OTBR z Laptopa Dell na platformę Raspberry Pi, w celu ograniczenia zużycia mocy oraz zapewnienia możliwość ulokowania urządzenia w dowolnym miejscu w docelowym budynku.
    \item Migracja serwisu Web GUI wraz z Bazą Danych oraz System Controllera do chmury, aby umożliwić użytkownikowi zdalną konfigurację parametrów systemu. Jako technologię chmurową zastosowana mogłaby zostać platforma \textit{Microsoft Azure}.
    \item Rozwinięcie Web GUI poprzez dodanie panelu, w którym umieszczone zostałyby dynamiczne generowane przebiegi czasowe z narzędzia Grafana.
\end{itemize}

Opracowanie sieci Thread oraz jej wdrożenie pozwoliły na przybliżenie charakterystyki stosu protokołów opracowanego przez Thread Group. Wykorzystanie mechanizmu Commissioning potwierdziło tezę o wysokim, w stosunku do innych protokołów sieci mesh (takich jak Bluetooth Mesh oraz Zigbee), poziomie bezpieczeństwa. Cecha samoleczenia oraz dynamiczne zarządzanie rolami i topologią sieci przez Lidera Thread, mityguje skutki awarii urządzeń w sieci. Integracja sieci domeny Thread z inną zewnętrzną siecią IP odbywa się poprzez włączenie do sieci urządzenia o zestandaryzowanej roli Rutera Brzegowego. Czyni to wdrażanie systemów IoT funkcjonujących z różnymi technologiami IP prostszym i skuteczniejszym. Takie zachowania stosu Thread mogą nieść szereg korzyści w zastosowaniach automatyki domowej i budynkowej. Dotyczy to szczególnie przypadku systemów, w których funkcjonuje duża liczba czujników oraz regulatorów, potrzebujących wymieniać pakiety z innymi, zewnętrznymi sieciami IP. Co więcej, możliwość integracji sieci Thread z Internetem, mogłaby ułatwić przeniesienie serwisów automatyki domowej lub budynkowej do chmury.

Analiza zużycia zasobów pamięci, w przypadku uruchomienia aplikacji działającej na stosie Thread, zwróciła uwagę na istotę estymacji zasobów platformy sprzętowej, przed przystąpieniem do tworzenia aplikacji dla systemów wbudowanych. Wymagania dotyczące wielkości RAM oraz ROM, podczas uruchomienia Thread oraz funkcjonalności z nim związanych, rzuciły cień na problem wdrożenia tego typu sieci w urządzeniach o małej liczbie pamięci ulotnej oraz nieulotnej.