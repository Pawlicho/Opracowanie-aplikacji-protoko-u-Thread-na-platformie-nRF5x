\chapter{Propozycja systemu}
\label{cha:propozycja-systemu}

Funkcją systemu jest utrzymanie zdefiniowanych w Tabeli 2.1 parametrów środowiska, na poziomie zadanym przez użytkownika systemu. W tym celu urządzenia systemu dokonują pomiaru parametrów oraz regulacji swojej mocy.

\begin{table}[H]
    \centering
    \caption{Zestaw parametrów określonych dla zdefiniowanych profili.}
    \begin{tabular}{|c|c|c|}
         \hline
         Profil & HEATER & DIMMER \\
         \hline
         Parametr & temperatura & natężenie oświetlenia \\
         \hline
         Jednostka & $^{\circ}$C & Lux \\
         \hline
    \end{tabular}
    \label{tab:my_label}
\end{table}

Na Rys. 2.1. przedstawiono system automatyki domowej, którego projekt i implementacja leży w zakresie kolejnych rozdziałów.

\begin{figure}[H]
    \centering
    \includegraphics[width=0.8\linewidth]{graphics/system-architecture.png}
    \caption{Proponowana architektura systemu automatyki domowej.}
    \label{fig:system-architecture}
\end{figure}

Zaproponowany system opiera się na architekturze klient-serwer i składa się z 5 funkcjonalnych bloków:
\begin{enumerate}
    \item klienta Dimmer,
    \item klienta Heater,
    \item serwera z aplikacją System Controller,
    \item bazy danych,
    \item Web GUI (ang. \textit{Graphical User Interface}).
\end{enumerate}

Celem niniejszego rozdziału jest omówienie zaproponowanego systemu oraz opisanie jego poszczególnych komponentów. 

\section{Strona kliencka}

Klientami w systemie są urządzenia połączone w sieć Thread, realizujące profile określone w Tabeli 2.1.. Rolą HEATER oraz DIMMER jest implementacja usług pomiaru i regulacji parametru.
Klienci komunikują się z serwerem w celu dostarczenia pomiarów oraz uzyskania informacji o decyzji dotyczącej konieczności regulacji mocy.
    
\section{Strona serwera}

System Controller to aplikacja znajdująca się w sieci spoza domeny Thread, pełniąca rolę serwera. Komponent przetwarza otrzymane wartości aktualnego stanu środowiska i uwzględniając zadane przez użytkownika wartości, dokonuje decyzji odnośnie regulacji. Ostatecznie aplikacja powiadamia klientów HEATER oraz DIMMER o konieczności zmiany mocy.

Dodatkowo System Controller odpowiedzialny jest za logowanie otrzymanych wartości aktualnego stanu parametrów oraz podjęte decyzje odnośnie regulacji.

\section{Sterowanie i Logowanie}

W celu zadania wartości parametrów, do których powinien dążyć system, użytkownik korzysta z Web GUI. Graficzny interfejs umożliwia zdefiniowanie stanu, jak również wgląd do logów systemu.

Wpisy systemowe oraz zadeklarowany stan parametrów użytkownika, przechowywane są w bazie danych, do której dostęp ma zarówno Web GUI, jak i System Controller.